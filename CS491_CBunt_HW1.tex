% Carl Bunt
% CS491
% Homework 1
\documentclass[a4paper, 11pt]{article}
\usepackage{comment} % enables the use of multi-line comments (\ifx \fi)
\usepackage{lipsum} %This package just generates Lorem Ipsum filler text.
\usepackage{fullpage} % changes the margin
\usepackage[a4paper, total={7in, 10in}]{geometry}
\usepackage[fleqn]{amsmath}
\usepackage{amssymb,amsthm}  % assumes amsmath package installed
\newtheorem{theorem}{Theorem}
\newtheorem{corollary}{Corollary}
\usepackage{graphicx}
\usepackage{tikz}
\usetikzlibrary{arrows}
\usepackage{verbatim}
\usepackage{float}
\usepackage{tikz}
\usetikzlibrary{shapes,arrows}
\usetikzlibrary{arrows,calc,positioning}

\tikzset{
  block/.style = {draw, rectangle,
    minimum height=1cm,
    minimum width=1.5cm},
  input/.style = {coordinate,node distance=1cm},
  output/.style = {coordinate,node distance=4cm},
  arrow/.style={draw, -latex,node distance=2cm},
  pinstyle/.style = {pin edge={latex-, black,node distance=2cm}},
  sum/.style = {draw, circle, node distance=1cm},
}
\usepackage{xcolor}
\usepackage{mdframed}
\usepackage[shortlabels]{enumitem}
\usepackage{indentfirst}
\usepackage{hyperref}
\hypersetup{
  colorlinks=true,
  linkcolor=red,
  filecolor=green,
  urlcolor=blue,
  }
\renewcommand{\thesubsection}{\thesection.\alph{subsection}}

\newenvironment{problem}[2][Question]
               { \begin{mdframed}[backgroundcolor=gray!20] \textbf{#1 #2:} \\}
               {   \end{mdframed}}

 % Define solution environment
 \newenvironment{response}
                {\textit{Response:}}
                {}

 \graphicspath{{./images/}{IR}}
 \newcommand*{\img}[2]{
   \begin{figure}[!ht]
     \flushleft
     \includegraphics[width=0.8\linewidth]{#1.JPG}
     \caption{\bf #2}
     \label{fig:#1}
 \end{figure}}
                
 \renewcommand{\qed}{\quad\qedsymbol}
 %%%%%%%%%%%%%%%%%%%%%%%%%%%%%%%%%%%%%%%%%%%%%%%%%
 \begin{document}
 % Header-Make sure you update this information!!!!
 \noindent
 %%%%%%%%%%%%%%%%%%%%%%%%%%%%%%%%%%%%%%%%%%%%%%%%%
 \large\textbf{Carl Bunt} \hfill \textbf{Homework - 1}   \\
 Email: cbunt@psu.edu \hfill ID: 915910838 \\
 \normalsize Course: CS 491 - Intro to Security \hfill Term: Fall 2020\\
 Instructor: Dr. Karen L. Karavanic \hfill Due Date: 07 OCT 20\\
 \noindent\rule{7in}{2.8pt}

 %%%%%%%%%%%%%%%%%%%%%%%%%%%%%%%%%%%%%%%%%%%%%%%%%
 % Problem 1
 %%%%%%%%%%%%%%%%%%%%%%%%%%%%%%%%%%%%%%%%%%%%%%%%%
 \begin{problem}{1}
   Written Exercises \\
   Provide 3 references to computer security that you find in the non-technical space.  Provide or describe what is said.  For each, say how it relates to any of the CIA:  confidentiality, integrity, availability.  For each, say how it relates to any of the STRIDE categories of threats.
   \begin{enumerate}[label=\alph*.]
   \item Examples might be fiction, tv shows, movies, magazines, blogs, even news.
   \item The reference may be technically correct or incorrect, or maybe you don’t know which.
   \item Include the source – movie name and year;  tv name, year, seasons\#, episode\#; URL, etc.  If possible, give a link.  A link to an entire movie is NOT useful although your graders would probably enjoy watching all of those movies.
   \end{enumerate}
 \end{problem}

 \begin{response}
   \begin{enumerate}[label=\arabic*)]
   \item Target Data Breach
     \begin{enumerate}[label=\alph*.]
     \item In late 2013 the credit card and personal data of 110 Million customers was stolen.
     \item How does this relate?
       \begin{itemize}
       \item[CIA:] The confidentiality of this data was compromised when it was made available to unauthorized people.
       \item[STRIDE:] While this was and made possible via tampering, it is an information disclosure at the heart of this event.
       \end{itemize}
     \item Wikipedia: \href{https://en.wikipedia.org/wiki/History_of_Target_Corporation#2013_security_breach}{Target Data breach}
     \end{enumerate}
   \item WannaCry
     \begin{enumerate}[label=\alph*.]
     \item The EternalBlue exploit allowed WannaCry to infect unpatched older machines to encrypt the system's data rendering it inaccessible to the user until they paid the ransom.
     \item How does this relate?
       \begin{itemize}
       \item[CIA:] This is largely an integrity issue and the only way to save yourself from permanent loss is to pay the ransom.
       \item[STRIDE:] WannaCry is threat of data tampering.
       \end{itemize}
     \item Wikipedia: \href{https://en.wikipedia.org/wiki/WannaCry_ransomware_attack}{WannaCry}
     \end{enumerate}
   \item Operation Payback
     \begin{enumerate}[label=\alph*.]
     \item A group of hacktivists in response to events they were not happy with recent events used a stress testing application to take control of a voluntary botnet and conduct a series of coordinated DDoS attacks.
     \item How does this relate?
       \begin{itemize}
       \item[CIA:] This attack directly hinders the availability of services provided by the victims of these attacks.
       \item[STRIDE:] This is a was a Threat to denial of service.
       \end{itemize}
     \item Wikipedia: \href{https://en.wikipedia.org/wiki/Operation_Payback}{Operation Payback}
     \end{enumerate}
   \end{enumerate}

 \end{response}
 \noindent\rule{7in}{2.8pt}

 %%%%%%%%%%%%%%%%%%%%%%%%%%%%%%%%%%%%%%%%%%%%%%%%%
 % Problem 2
 %%%%%%%%%%%%%%%%%%%%%%%%%%%%%%%%%%%%%%%%%%%%%%%%%
 \begin{problem}{2} 

   Draw an attack tree that describes a cheating student obtaining the answers to another student’s homework.  Include all paths you can think of.  You may do this electronically, or draw on paper then take a picture with your phone.
 \end{problem}

 \begin{response}
   Homework cheating attack tree \\
   \img{Threat-tree}{Homework attack tree}
 \end{response}
 \noindent\rule{7in}{2.8pt}
 \end{document}
