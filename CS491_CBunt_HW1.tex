% Carl Bunt
% CS491
% Homework 1
\documentclass[a4paper, 11pt]{article}
\usepackage{comment} % enables the use of multi-line comments (\ifx \fi)
\usepackage{lipsum} %This package just generates Lorem Ipsum filler text.
\usepackage{fullpage} % changes the margin
\usepackage[a4paper, total={7in, 10in}]{geometry}
\usepackage[fleqn]{amsmath}
\usepackage{amssymb,amsthm}  % assumes amsmath package installed
\newtheorem{theorem}{Theorem}
\newtheorem{corollary}{Corollary}
\usepackage{graphicx}
\usepackage{tikz}
\usetikzlibrary{arrows}
\usepackage{verbatim}
\usepackage{float}
\usepackage{tikz}
\usetikzlibrary{shapes,arrows}
\usetikzlibrary{arrows,calc,positioning}

\tikzset{
  block/.style = {draw, rectangle,
    minimum height=1cm,
    minimum width=1.5cm},
  input/.style = {coordinate,node distance=1cm},
  output/.style = {coordinate,node distance=4cm},
  arrow/.style={draw, -latex,node distance=2cm},
  pinstyle/.style = {pin edge={latex-, black,node distance=2cm}},
  sum/.style = {draw, circle, node distance=1cm},
}
\usepackage{xcolor}
\usepackage{mdframed}
\usepackage[shortlabels]{enumitem}
\usepackage{indentfirst}
\usepackage{hyperref}

\renewcommand{\thesubsection}{\thesection.\alph{subsection}}

\newenvironment{problem}[2][Question]
               { \begin{mdframed}[backgroundcolor=gray!20] \textbf{#1 #2:} \\}
               {  \end{mdframed}}

 % Define solution environment
 \newenvironment{response}
                {\textit{Response:}}
                {}

 \renewcommand{\qed}{\quad\qedsymbol}
 %%%%%%%%%%%%%%%%%%%%%%%%%%%%%%%%%%%%%%%%%%%%%%%%%
 \begin{document}
 % Header-Make sure you update this information!!!!
 \noindent
 %%%%%%%%%%%%%%%%%%%%%%%%%%%%%%%%%%%%%%%%%%%%%%%%%
 \large\textbf{Carl Bunt} \hfill \textbf{Homework - 1}   \\
 Email: cbunt@psu.edu \hfill ID: 915910838 \\
 \normalsize Course: CS 491 - Intro to Security \hfill Term: Fall 2020\\
 Instructor: Dr. Karen L. Karavanic \hfill Due Date: 07 OCT 20\\
 \noindent\rule{7in}{2.8pt}

 %%%%%%%%%%%%%%%%%%%%%%%%%%%%%%%%%%%%%%%%%%%%%%%%%
 % Problem 1
 %%%%%%%%%%%%%%%%%%%%%%%%%%%%%%%%%%%%%%%%%%%%%%%%%
 \begin{problem}{1}
   Provide 3 references to computer security that you find in the non-technical space.  Provide or describe what is said.  For each, say how it relates to any of the CIA:  confidentiality, integrity, availability.  For each, say how it relates to any of the STRIDE categories of threats.
   \begin{enumerate}[label=\alpha*.]
   \item Examples might be fiction, tv shows, movies, magazines, blogs, even news.
   \item The reference may be technically correct or incorrect, or maybe you don’t know which.
   \item Include the source – movie name and year;  tv name, year, seasons#, episode \#; URL, etc.  If possible, give a link.  A link to an entire movie is NOT useful although your graders would probably enjoy watching all of those movies.
   \end{enumerate}
 \end{problem}

 \begin{response}
 \end{response}
 \noindent\rule{7in}{2.8pt}

 %%%%%%%%%%%%%%%%%%%%%%%%%%%%%%%%%%%%%%%%%%%%%%%%%
 % Problem 2
 %%%%%%%%%%%%%%%%%%%%%%%%%%%%%%%%%%%%%%%%%%%%%%%%% 
 \begin{problem}{2}
   Draw an attack tree that describes a cheating student obtaining the answers to another student’s homework.  Include all paths you can think of.  You may do this electronically, or draw on paper then take a picture with your phone.
 \end{problem}
 
 \begin{response}
 \end{response}
 \noindent\rule{7in}{2.8pt}
