% Carl Bunt
% CS491
% Homework 4
\documentclass[a4paper, 11pt]{article}
\usepackage{comment} % enables the use of multi-line comments (\ifx \fi)
\usepackage{lipsum} %This package just generates Lorem Ipsum filler text.
\usepackage{fullpage} % changes the margin
\usepackage[a4paper, total={7in, 10in}]{geometry}
\usepackage[fleqn]{amsmath}
\usepackage{amssymb,amsthm}  % assumes amsmath package installed
\newtheorem{theorem}{Theorem}
\newtheorem{corollary}{Corollary}
\usepackage{graphicx}
\usepackage{tikz}
\usetikzlibrary{arrows}
\usepackage{verbatim}
\usepackage{float}
\usepackage{tikz}
\usepackage{listings}
\usepackage{multicol}
\usetikzlibrary{shapes,arrows}
\usetikzlibrary{arrows,calc,positioning}

\tikzset{
  block/.style = {draw, rectangle,
    minimum height=1cm,
    minimum width=1.5cm},
  input/.style = {coordinate,node distance=1cm},
  output/.style = {coordinate,node distance=4cm},
  arrow/.style={draw, -latex,node distance=2cm},
  pinstyle/.style = {pin edge={latex-, black,node distance=2cm}},
  sum/.style = {draw, circle, node distance=1cm},
}

\usepackage{xcolor}
\usepackage{mdframed}
\usepackage[shortlabels]{enumitem}
\usepackage{indentfirst}
\usepackage{hyperref}
\hypersetup{
  colorlinks=true,
  linkcolor=red,
  filecolor=green,
  urlcolor=blue,
  }
\definecolor{codegreen}{rgb}{0,0.6,0}
\definecolor{codegray}{rgb}{0.5,0.5,0.5}
\definecolor{codeblue}{rgb}{0,0,0.8}
\definecolor{backcolour}{rgb}{0.95,0.95,0.95}

\lstdefinestyle{C-code}{
  backgroundcolor=\color{backcolour},
  commentstyle=\color{codegreen},
  keywordstyle=\color{codeblue},
  numberstyle=\tiny\color{codegray},
  stringstyle=\color{magenta},
  basicstyle=\ttfamily\footnotesize,
  breakatwhitespace=false,
  breaklines=true,
  captionpos=b,
  keepspaces=true,
  numbers=left,
  numbersep=5pt,
  showspaces=false,
  showstringspaces=false,
  showtabs=false,
  tabsize=2
  }
\lstset{style=c-code}
\renewcommand{\thesubsection}{\thesection.\alph{subsection}}

\newenvironment{problem}[2][Question]
               { \begin{mdframed}[backgroundcolor=gray!20] \textbf{#1 #2:} \\}
               {   \end{mdframed}}

 % Define solution environment
 \newenvironment{response}
                {\textit{Response:}}
                {}

 \graphicspath{{./images/}{IR}}
 \newcommand*{\img}[2]{
   \begin{figure}[!ht]
     \flushleft
     \includegraphics[width=0.8\linewidth]{#1.JPG}
     \caption{\bf #2}
     \label{fig:#1}
 \end{figure}}
                
 \renewcommand{\qed}{\quad\qedsymbol}
 %%%%%%%%%%%%%%%%%%%%%%%%%%%%%%%%%%%%%%%%%%%%%%%%%
 \begin{document}
 % Header-Make sure you update this information!!!!
 \noindent
 %%%%%%%%%%%%%%%%%%%%%%%%%%%%%%%%%%%%%%%%%%%%%%%%%
 \large\textbf{Carl Bunt} \hfill \textbf{Homework - 4}   \\
 Email: cbunt@psu.edu \hfill ID: 915910838 \\
 \normalsize Course: CS 491 - Intro to Security \hfill Term: Fall 2020\\
 Instructor: Dr. Karen L. Karavanic \hfill Due Date: 18 NOV 20\\
 \noindent\rule{7in}{2.8pt}

 %%%%%%%%%%%%%%%%%%%%%%%%%%%%%%%%%%%%%%%%%%%%%%%%%
 % Problem A
 %%%%%%%%%%%%%%%%%%%%%%%%%%%%%%%%%%%%%%%%%%%%%%%%%
 \begin{problem}{A}
   Written Exercises \\
   \begin{enumerate}[label=\arabic*.]
   \item
     Define: Race condition     
   \item
     Define: Atomic operation
   \item
     Define: Impact surface
   \item
     Dectribe the attack surface for the Linux Lab machies in CS.
   \end{enumerate}
 \end{problem}

 \begin{response}
   \begin{enumerate}[label=\arabic*.]
   \item
     \bf{Race condition}: When 2 threads are attempting to access data at the same time and there is no way to guarantee which will get to it first.
   \item
     \bf{Atomic operation}: An operation that is compleated in a single process cycle.
   \item
     \bf{Impact surface}: The collection all events that happen as a result of a successful exploit.
   \item
     \bf{Linux Lab attack surface}: The three most obvious attack vectors I can think of are:
     \begin{itemize}[leftmargin=100pt]
     \item[Physical security]
       A nefarious actor may obtian physical access to the system to steal physical system to access private data.
     \item[Network access]
       A nefarious actor may either on local network or on across the internet send commands to invoke maliscous code.
     \item[User ignorance]
       A User may fall victim of a phishing scheme or introduce an infected file to the system.
     \end{itemize}  
   \end{enumerate}      
 \end{response}
 \noindent\rule{7in}{2.8pt}
\newpage
 %%%%%%%%%%%%%%%%%%%%%%%%%%%%%%%%%%%%%%%%%%%%%%%%%
 % Problem B
 %%%%%%%%%%%%%%%%%%%%%%%%%%%%%%%%%%%%%%%%%%%%%%%%%
 \begin{problem}{B} 
   Hands On: Linux/C/C++\\

   \begin{enumerate}[label=\arabic*.]
   \item
     Submit an 8-by-8 table of the conversion effects of in C or C++ of data types.
   \item
     Provide one example of exploitable code for each of integer overflow and integer underflow. 
   \item
     write a C/C++ program with a buffer overflow vulnerability.
      
  \end{enumerate}
\end{problem}

\begin{response}
  \begin{enumerate}[label=\arabic*.]
  \item
    C/C++ conversions
    \begin{table}[h!]
      \label{tab:Conversions} {\tiny
      \begin{tabular}{|l|l|l|l|l|l|l|l|l|}
        \hline 
                 & s-char     & u-char     & s-short    & u-short    & s-int      & u-int      & s-long     & u-long    \\ \hline
        s-char   & SWID BP VP & SWID BP VC & ZEXT BC VP & ZEXT BC VC & ZEXT BC VP & ZEXT BC VC & ZEXT BC VP & ZEXT BC VC\\ \hline
        u-char   & SWID BP VC & SWID BP VP & SEXT BC VP & SEXT BC VP & SEXT BC VP & SEXT BC VP & SEXT BC VP & SEXT BC VP\\ \hline
        s-short  & TRNC BP VC & TRNC BP VC & SWID BP VP & SWID BP VC & ZEXT BC VP & ZEXT BC VC & ZEXT BC VP & ZEXT BC VC\\ \hline
        u-short  & TRNC BP VC & TRNC BP VC & SWID BP VC & SWID BP VP & SEXT BC VP & SEXT BC VP & SEXT BC VP & SEXT BC VP\\ \hline
        s-int    & TRNC BP VC & TRNC BP VC & TRNC BP VC & TRNC BP VC & SWID BP VP & SWID BP VC & ZEXT BC VP & ZEXT BC VC\\ \hline
        u-int    & TRNC BP VC & TRNC BP VC & TRNC BP VC & TRNC BP VC & SWID BP VC & SWID BP VP & SEXT BC VP & SEXT BC VP\\ \hline
        s-long   & TRNC BP VC & TRNC BP VC & TRNC BP VC & TRNC BP VC & TRNC BP VC & TRNC BP VC & SWID BP VP & SWID BP VC\\ \hline
        u-long   & TRNC BP VC & TRNC BP VC & TRNC BP VC & TRNC BP VC & TRNC BP VC & TRNC BP VC & SWID BP VC & SWID BP VP\\ \hline
      \end{tabular}}
      \caption{Data Type Conversions
          \\{\bf Key:}
          \\sign-extended(SEXT) \\zero-extended(ZEXT) \\same width(SWID)
          \\truncated(TRNC) \\value changed(VC) \\value preserved(VP)
          \\bit-pattern changed(BC) \\bit-pattern preserved(BP)}
    \end{table}
  \item
    Integer OverFlow
    \lstinputlisting[language=C++]{OverFlow.c}
    Say you buy a bulk order of shirts for \$20 each and the total is stored in a 16-bit uint. All is fine unless you order more then 3277. The overflow will reslut in a order total of \$4.
    \newpage
    Integer UnderFlow
    \lstinputlisting[language=C++]{UnderFlow.c}
    Now you are returning the exess shirts. This time it is stored in a 16-bit int. Account debt will show in a negative value until 1639 shirts are returned then suddenly you will owe them \$32756 to return those shirts.
    
  \item
    Buffer overFlow
    \lstinputlisting[language=C++]{BufferOverFlow.c}
    If running the code with char string  of 8 characters then the total will be \$0. Anything longer then that the total is over writen with the ASCII input.

    \begin{table}[h!]
      \centering
      \label{tab:Buffer OverFlow Results}
      \begin{tabular}{|l|r|}
        \hline
        ./BufferOverFlow 10 Carl & \$200 \\ \hline
        ./BufferOverFlow 10 Carl0000 & \$0 \\ \hline
        ./BufferOverFlow 10 Carl00000 & \$48 \\ \hline
        ./BufferOverFlow 10 Carl000000 & \$12336 \\ \hline
      \end{tabular}
    \end{table}
  \end{enumerate}
\end{response}
\noindent\rule{7in}{2.8pt}
\end{document}
